\documentclass[11pt]{article}
\usepackage{geometry}
\geometry{a4paper, margin=1in}

\begin{document}

\title{Location Tracker Semi-Structured Interview Protocol}
\author{}
\date{February 2026}

\maketitle
\section{Introduction}
The purpose of a semi-structured interview is to gather additional information from the user that may not have been captured by geolocation or diary entries. This information will be used to supplement the study data and to investigate any relationship between participation (as measured by questionnaire) and the geolocation based summary measures.

The session will provide an opportunity to discuss the participant's experience of the app, their perception of participating in the study and to reiterate how their data will be used.

\subsection{Study Feedback}
The participant will be asked to provide feedback on the study, including:
\begin{itemize}
    \item In your own words, what was your overall experience of participating in the study? You may include positive and negative aspects of the study, as well as any suggestions for improvement in your answer.
    \item What was your experience of having the types of locations you visited during this period recorded during this study?
        \begin{itemize}
            \item How do you feel about it now?
            \item Do you have any other comments or questions about the data collected during this study, or what is done with it?
        \end{itemize}
    \item What is your understanding of the data you have provided and how it will be used.
    \item Do you feel your participation and activity over the data collection period was typical for you?
\end{itemize}

\section{Participation questionnaire content}
The semi-structured interview is an opportunity to ask participants questions that align with one, or more, of the questionnaire items from the following measures of participation (The rationale for inclusion of each measure is included):
    
\begin{enumerate}
        \item \textbf{WHODAS 2.0} - Full alignment with ICF and accepted measure of participation in a range of demographics. (ONLY MEASURE APPLICABLE TO ALL PARTICIPANTS)
        \item  \textbf{YADAPS} - Measure aimed at young adults, captures elemetns of online participation
        \item  \textbf{USER-Participation} - Similar to above, clearer frequency based categories and not specifically for YA.
        \item \textbf{PM-PAC} - Some open ended questions that may capture other elements of participation.
        \item  \textbf{SF36-V2} - Interesting metrics on mental well-being.
        \item  \textbf{GCPLA-r} - Tangible activities (swimming, shopping gp visits etc.) and short form, so easily administered.
    \end{enumerate}

\end{document}